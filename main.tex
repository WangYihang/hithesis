% !Mode:: "TeX:UTF-8"
\documentclass[
    newtxmath=true,
    newgeometry=two,
    capcenterlast=true,
    subcapcenterlast=true,
    openright=false,
    library=false,
    absupper=true,
    fontset=windowsnew,
    type=bachelor,
    campus=harbin,
    stage=opening
]{hithesis}
% 此处选项中不要有空格
%%%%%%%%%%%%%%%%%%%%%%%%%%%%%%%%%%%%%%%%%%%%%%%%%%%%%%%%%%%%%%%%%%%%%%%%%%%%%%%%
% 必填选项
% type=doctor|master|bachelor
%%%%%%%%%%%%%%%%%%%%%%%%%%%%%%%%%%%%%%%%%%%%%%%%%%%%%%%%%%%%%%%%%%%%%%%%%%%%%%%%
% 选填选项(选填选项的缺省值已经尽可能满足了大多数需求,除非明确知道自己有什么
% 需求)
% campus=shenzhen|weihai|harbin
%   含义:校区选项,默认harbin
% stage=opening|interim|final
%   含义:毕业设计的阶段,在修改为当前阶段后,需要在本文下方的正文区域寻找对应的 stage 并取消注释
%       opening 开题
%       interim 中期
%       final 结题
% glue=true|false
%   含义:由于我工规范中要求字体行距在一个闭区间内,这个选项为true表示tex自
%   动选择,为false表示区间内一个最接近版心要求行数的要求的默认值,缺省值为
%   false。
% tocfour=true|false
%   含义:是否添加第四级目录,只对本科文科个别要求四级目录有效,缺省值为
%   false
% fontset=siyuan|windowsnew|windowsold
%   含义:注意这个选项视为了解决特殊问题而设置,比如用有些发行版本的linux排
%   版时可能(大多数发行版不会)会遇到的字体无法载入的问题,或者字体载入之
%   后出现无法复制的问题以及想要解决排版如 biang biang 面的 biang 这类中易
%   宋体无法识别的汉字的问题。没有特殊的需要不推荐使用这个选项。
%   
%   如果是安装了 windows 字体的 linux 系统,可以填写windowsnew(win vista
%   以后 的字体)或 windowsold(vista 以前)或者想用思源宋体并且是已经安装
%   了思源宋体的任何系统,填写siyuan选项。缺省值为空,自动识别系统并匹配字体
%   。模板版中给出的思源字体定义文件定义的思源字体的版本是Adobe版,其他字体
%   是windowsnew字体。
% tocblank=true|false
%   含义:目录中第一章之前,是否加一行空白。缺省值为true。
% chapterhang=true|false
%   含义:目录的章标题是否悬挂居中,规范中要求章标题少于15字,所以这个选项
%   有无没什么用,除了特殊需求。缺省值为true。
% fulltime=true|false
%   含义:是否全日制,缺省值为true。非全日制如同等学力等,要在cover中设置类
%   型,封面中不同格式
% subtitle=true|false
%   含义:论文题目是否含有副标题,缺省值为false,如果有要在cover中设置副标
%   题内容,封面中显示。
% newgeometry=one|two
%   含义:规范中的自相矛盾之处,版芯是否包含页眉页脚,旧方法是按照包含页眉
%   页脚来设置。该选项是多选选项,如果没有这个选项,缺省值是旧模板的版芯设
%   置方法,如果设置该选项one或two,分别对应两种页眉页码对应版芯线的相对位
%   置。第一种是严格按照规范要求,难看。第二种微调了页眉页码位置,好一点。
% debug=true|false
%   含义:是否显示版芯框和行号,用来调试。默认否。
% openright=true|false
%   含义:博士论文是否要求章节首页必须在奇数页,此选项不在规范要求中,按个
%   人喜好自行决定。 默认否。注意,窝工的默认情况是打印版博士论文要求右翻页
%   ,电子版要求非右翻页且无空白页。如果想DIY(或身不由己DIY)在什么地方右
%   翻页,将这个选项设置为false,然后在目标位置添加`\cleardoublepage`命令即
%   可。
% library=true|false
%   含义:是否为提交到图书馆的电子版。默认否。注意:如果设置成true,那么
%   openright选项将被强制转换为false。
% capcenterlast=true|false
%   含义:图题、表题最后一行是否居中对齐(我工规范要求居中,但不要求居中对
%   齐),此选项不在规范要求中,按个人喜好自行决定。默认否。
% subcapcenterlast=true|false
%   含义:子图图题最后一行是否居中对齐(我工规范要求居中,但不要求居中对齐
%   ),此选项不在规范要求中,按个人喜好自行决定。默认否。
% absupper=true|false
%   含义:中文目录中的英文摘要在中文目录中的大小写样式歧义,在规范中要求首
%   字母大写,在work样例中是全大写。该选项控制是否全大写。默认否。
% bsmainpagenumberline=true|false
%   含义:由于本科生论文官方模板的页码和页眉格式混乱,提供这个选项自定义设
%   置是否在正文中显示页码横线,默认否。
% bsfrontpagenumberline=true|false
%   含义:由于本科生论文官方模板的页码和页眉格式混乱,提供这个选项自定义设
%   置是否在前文中显示页码横线,默认否。
% bsheadrule=true|false
%   含义:由于本科生论文官方模板的页码和页眉格式混乱,提供这个选项自定义设
%   置是否显示页眉横线,默认显示。
% splitbibitem=true|false
%   含义:参考文献每一个条目内能不能断页,应广大刀客要求添加。默认否。
% newtxmath=true|false
%   含义:数学字体是否使用新罗马。默认是。
%%%%%%%%%%%%%%%%%%%%%%%%%%%%%%%%%%%%%%%%%%%%%%%%%%%%%%%%%%%%%%%%%%%%%%%%%%%%%%%%

\usepackage{hithesis}

% 定义封面信息
% !Mode:: "TeX:UTF-8"

\hitsetup{
  %******************************
  % 注意:
  %   1. 配置里面不要出现空行
  %   2. 不需要的配置信息可以删除
  %******************************
  ctitleone={局部多孔质气体静压},%本科生封面使用
  ctitletwo={轴承关键技术的研究},%本科生封面使用
  ctitlecover={局部多孔质气体静压轴承关键技术的研究},%放在封面中使用,自由断行
  ctitle={局部多孔质气体静压轴承关键技术的研究},%放在原创性声明中使用
  csubtitle={一条副标题}, %一般情况没有,可以注释掉
  cxueke={工学},
  csubject={机械制造及其自动化},
  caffil={机电工程学院},
  cauthor={于冬梅},
  csupervisor={某某某教授},
  % 日期自动使用当前时间,若需指定按如下方式修改:
  cdate={2024年04月01日},
  cstudentid={1200300101},
}

% 定义所包含图片根目录
\graphicspath{{figures/}}
% 物理量名称表,学院有要求添加
% \input{front/denotation}


\begin{document}
%%%%%%%%%%%%%%%%%%%%%%%%%%%%%%%%%%%
% 本科生开题答辩
%%%%%%%%%%%%%%%%%%%%%%%%%%%%%%%%%%%
\frontmatter
% 生成封面
\makecover
% 生成目录
\tableofcontents

\mainmatter
% 根据学校教务处开题报告模版将内容分为如下几个章节
\chapter{课题来源及研究的目的和意义}

\section{课题来源}
本课题来源于...项目。

\section{研究的目的和意义}
研究的目的和意义有...。

\chapter{国内外在该方向的研究现状及分析}

国内外在该方向的研究现状如下...。
\chapter{主要研究内容}

主要研究内容为...。
\chapter{研究方案}

研究方案为...。
\chapter{进度安排与预期达到的目标}

\section{进度安排}

\section{预期达到的目标}
预期达到的目标是...。

\chapter{课题已具备和所需的条件、经费}

课题已具备和所需的条件、经费有...。
\chapter{研究过程中可能遇到的困难和问题,解决的措施}

研究过程中可能遇到的困难和问题,解决的措施有...。

\backmatter
\bibliographystyle{hithesis}
\bibliography{reference}

% %%%%%%%%%%%%%%%%%%%%%%%%%%%%%%%%%%%
% % 本科生中期答辩
% %%%%%%%%%%%%%%%%%%%%%%%%%%%%%%%%%%%
% \frontmatter
% % 生成封面
% \makecover
% % 生成目录
% \tableofcontents

% \mainmatter
% % 根据学校教务处中期报告模版将内容分为如下几个章节
% % !Mode:: "TeX:UTF-8"

\chapter{课题研究内容}

本课题来源于...项目。
感谢前辈们 \cite{Chen1992} 的贡献。
% \chapter{研究方案}

本课题采用...的研究方案。

% \chapter{目前已完成的研究工作及成果}

% \section{开题时工作安排时间表}
% 开题时,对毕业设计的进度与时间节点安排如表 \ref{tab:schedule} 所示,\emph{目前正在按照计划进行}。
% \begin{table}[htbp]
%     \centering
%     \bicaption{}{开题时工作安排时间表}{Table$\!$}{Schedule in opening report}\vspace{0.5em}\wuhao
%     \begin{tabularx}{\textwidth}{lX}
%     \toprule[1.5pt]
%     预计时间 & 计划完成的工作 \\
%     \midrule[1pt]
%     2024.01 & 工作 1 \\
%     2024.02 & 工作 2 \\
%     2024.03 & 工作 3 \\
%     2024.03 & 工作 4 \\
%     2024.04 & 工作 5 \\
%     2024.05 & 工作 6 \\
%     2024.06 & 工作 7 \\
%     \bottomrule[1.5pt]
%     \label{tab:schedule}
%     \end{tabularx}
% \end{table}

下面对已完成的成果进行介绍。
% \chapter{后期拟完成的研究工作与进度安排}\label{ch:future_work}

\section{后期拟完成的研究工作}
    \begin{enumerate}
        \item 工作 1
        \item 工作 2
        \item 工作 3
        \item 工作 4
        \item 工作 5
        \item 工作 6
    \end{enumerate}

\section{后期工作时间安排表}
    \begin{table}[htbp]
        \centering
        \bicaption{}{后期工作安排时间表}{Table$\!$}{Schedule in future}\vspace{0.5em}\wuhao
        \begin{tabularx}{\textwidth}{llX}
        \toprule[1.5pt]
        预计时间 & 周数 & 计划完成的工作 \\
        \midrule[1pt]
        2024.03 & 2 & 工作 1 \\
        2024.03 & 2 & 工作 2 \\
        2024.04 & 2 & 工作 3 \\
        2024.04 & 2 & 工作 4 \\
        2024.05 & 2 & 工作 5 \\
        2024.06 & 2 & 工作 6 \\
        \bottomrule[1.5pt]
        \label{tab:futureschedule}
        \end{tabularx}
    \end{table}




% \chapter{存在的困难与问题}

\begin{enumerate}
    \item 困难与问题 1
    \item 困难与问题 2
    \item 困难与问题 3
\end{enumerate}
% \chapter{如期完成全部论文工作的可能性}

可以如期完成

% \backmatter
% \bibliographystyle{hithesis}
% \bibliography{reference}

%%%%%%%%%%%%%%%%%%%%%%%%%%%%%%%%%%%
% 本科生毕业设计(论文)
%%%%%%%%%%%%%%%%%%%%%%%%%%%%%%%%%%%
% \frontmatter
% % 生成封面
% \makecover
% % 生成目录
% \tableofcontents

% \mainmatter
% % 根据学校教务处中期报告模版将内容分为如下几个章节
% \chapter{绪论}

% \chapter{静态污点分析技术}

% \chapter{动态污点分析技术}

% \chapter{系统设计与实现}

% \chapter{实验设计}

% \chapter{实验结果分析}


% \backmatter
% % !Mode:: "TeX:UTF-8" 
\begin{conclusions}

学位论文的结论作为论文正文的最后一章单独排写,但不加章标题序号。

结论应是作者在学位论文研究过程中所取得的创新性成果的概要总结,不能与摘要混为一谈。博士学位论文结论应包括论文的主要结果、创新点、展望三部分,在结论中应概括论文的核心观点,明确、客观地指出本研究内容的创新性成果(含新见解、新观点、方法创新、技术创新、理论创新),并指出今后进一步在本研究方向进行研究工作的展望与设想。对所取得的创新性成果应注意从定性和定量两方面给出科学、准确的评价,分(1)、(2)、(3)…条列出,宜用“提出了”、“建立了”等词叙述。

\end{conclusions}
   % 结论
% \bibliographystyle{hithesis}
% \bibliography{reference}
% \authorization %授权
% % \authorization[saomiao.pdf] %添加扫描页的命令,与上互斥
% % !Mode:: "TeX:UTF-8"
\begin{acknowledgements}

衷心感谢导师~XXX~教授对本人的精心指导。他的言传身教将使我终生受益。

\end{acknowledgements}
 %致谢
% % \begin{appendix}%附录
% %     \chapter{外文资料原文}
\label{cha:engorg}

\title{The title of the English paper}

\textbf{Abstract:} As one of the most widely used techniques in operations
research, \emph{ mathematical programming} is defined as a means of maximizing a
quantity known as \emph{bjective function}, subject to a set of constraints
represented by equations and inequalities. Some known subtopics of mathematical
programming are linear programming, nonlinear programming, multiobjective
programming, goal programming, dynamic programming, and multilevel
programming$^{[1]}$.

It is impossible to cover in a single chapter every concept of mathematical
programming. This chapter introduces only the basic concepts and techniques of
mathematical programming such that readers gain an understanding of them
throughout the book$^{[2,3]}$.


\section{Single-Objective Programming}
The general form of single-objective programming (SOP) is written
as follows,
\begin{equation}\tag*{(123)} % 如果附录中的公式不想让它出现在公式索引中,那就请
                             % 用 \tag*{xxxx}
\left\{\begin{array}{l}
\max \,\,f(x)\\[0.1 cm]
\mbox{subject to:} \\ [0.1 cm]
\qquad g_j(x)\le 0,\quad j=1,2,\cdots,p
\end{array}\right.
\end{equation}
which maximizes a real-valued function $f$ of
$x=(x_1,x_2,\cdots,x_n)$ subject to a set of constraints.

\newtheorem{mpdef}{Definition}[chapter]
\begin{mpdef}
In SOP, we call $x$ a decision vector, and
$x_1,x_2,\cdots,x_n$ decision variables. The function
$f$ is called the objective function. The set
\begin{equation}\tag*{(456)} % 这里同理,其它不再一一指定。
S=\left\{x\in\Re^n\bigm|g_j(x)\le 0,\,j=1,2,\cdots,p\right\}
\end{equation}
is called the feasible set. An element $x$ in $S$ is called a
feasible solution.
\end{mpdef}

\newtheorem{mpdefop}[mpdef]{Definition}
\begin{mpdefop}
A feasible solution $x^*$ is called the optimal
solution of SOP if and only if
\begin{equation}
f(x^*)\ge f(x)
\end{equation}
for any feasible solution $x$.
\end{mpdefop}

One of the outstanding contributions to mathematical programming was known as
the Kuhn-Tucker conditions\ref{eq:ktc}. In order to introduce them, let us give
some definitions. An inequality constraint $g_j(x)\le 0$ is said to be active at
a point $x^*$ if $g_j(x^*)=0$. A point $x^*$ satisfying $g_j(x^*)\le 0$ is said
to be regular if the gradient vectors $\nabla g_j(x)$ of all active constraints
are linearly independent.

Let $x^*$ be a regular point of the constraints of SOP and assume that all the
functions $f(x)$ and $g_j(x),j=1,2,\cdots,p$ are differentiable. If $x^*$ is a
local optimal solution, then there exist Lagrange multipliers
$\lambda_j,j=1,2,\cdots,p$ such that the following Kuhn-Tucker conditions hold,
\begin{equation}
\label{eq:ktc}
\left\{\begin{array}{l}
    \nabla f(x^*)-\sum\limits_{j=1}^p\lambda_j\nabla g_j(x^*)=0\\[0.3cm]
    \lambda_jg_j(x^*)=0,\quad j=1,2,\cdots,p\\[0.2cm]
    \lambda_j\ge 0,\quad j=1,2,\cdots,p.
\end{array}\right.
\end{equation}
If all the functions $f(x)$ and $g_j(x),j=1,2,\cdots,p$ are convex and
differentiable, and the point $x^*$ satisfies the Kuhn-Tucker conditions
(\ref{eq:ktc}), then it has been proved that the point $x^*$ is a global optimal
solution of SOP.

\subsection{Linear Programming}
\label{sec:lp}

If the functions $f(x),g_j(x),j=1,2,\cdots,p$ are all linear, then SOP is called
a {\em linear programming}.

The feasible set of linear is always convex. A point $x$ is called an extreme
point of convex set $S$ if $x\in S$ and $x$ cannot be expressed as a convex
combination of two points in $S$. It has been shown that the optimal solution to
linear programming corresponds to an extreme point of its feasible set provided
that the feasible set $S$ is bounded. This fact is the basis of the {\em simplex
  algorithm} which was developed by Dantzig as a very efficient method for
solving linear programming.
\begin{table}[ht]
\centering
  \centering
  \caption*{Table~1\hskip1em This is an example for manually numbered table, which
    would not appear in the list of tables}
  \label{tab:badtabular2}
  \begin{tabular}[c]{|m{1.5cm}|c|c|c|c|c|c|}\hline
    \multicolumn{2}{|c|}{Network Topology} & \# of nodes &
    \multicolumn{3}{c|}{\# of clients} & Server \\\hline
    GT-ITM & Waxman Transit-Stub & 600 &
    \multirow{2}{2em}{2\%}&
    \multirow{2}{2em}{10\%}&
    \multirow{2}{2em}{50\%}&
    \multirow{2}{1.2in}{Max. Connectivity}\\\cline{1-3}
    \multicolumn{2}{|c|}{Inet-2.1} & 6000 & & & &\\\hline
    & \multicolumn{2}{c|}{ABCDEF} &\multicolumn{4}{c|}{} \\\hline
\end{tabular}
\end{table}

Roughly speaking, the simplex algorithm examines only the extreme points of the
feasible set, rather than all feasible points. At first, the simplex algorithm
selects an extreme point as the initial point. The successive extreme point is
selected so as to improve the objective function value. The procedure is
repeated until no improvement in objective function value can be made. The last
extreme point is the optimal solution.

\subsection{Nonlinear Programming}

If at least one of the functions $f(x),g_j(x),j=1,2,\cdots,p$ is nonlinear, then
SOP is called a {\em nonlinear programming}.

A large number of classical optimization methods have been developed to treat
special-structural nonlinear programming based on the mathematical theory
concerned with analyzing the structure of problems.

Now we consider a nonlinear programming which is confronted solely with
maximizing a real-valued function with domain $\Re^n$.  Whether derivatives are
available or not, the usual strategy is first to select a point in $\Re^n$ which
is thought to be the most likely place where the maximum exists. If there is no
information available on which to base such a selection, a point is chosen at
random. From this first point an attempt is made to construct a sequence of
points, each of which yields an improved objective function value over its
predecessor. The next point to be added to the sequence is chosen by analyzing
the behavior of the function at the previous points. This construction continues
until some termination criterion is met. Methods based upon this strategy are
called {\em ascent methods}, which can be classified as {\em direct methods},
{\em gradient methods}, and {\em Hessian methods} according to the information
about the behavior of objective function $f$. Direct methods require only that
the function can be evaluated at each point. Gradient methods require the
evaluation of first derivatives of $f$. Hessian methods require the evaluation
of second derivatives. In fact, there is no superior method for all
problems. The efficiency of a method is very much dependent upon the objective
function.

\subsection{Integer Programming}

{\em Integer programming} is a special mathematical programming in which all of
the variables are assumed to be only integer values. When there are not only
integer variables but also conventional continuous variables, we call it {\em
  mixed integer programming}. If all the variables are assumed either 0 or 1,
then the problem is termed a {\em zero-one programming}. Although integer
programming can be solved by an {\em exhaustive enumeration} theoretically, it
is impractical to solve realistically sized integer programming problems. The
most successful algorithm so far found to solve integer programming is called
the {\em branch-and-bound enumeration} developed by Balas (1965) and Dakin
(1965). The other technique to integer programming is the {\em cutting plane
  method} developed by Gomory (1959).

\hfill\textit{Uncertain Programming\/}\quad(\textsl{BaoDing Liu, 2006.2})

\section*{References}
\noindent{\itshape NOTE: These references are only for demonstration. They are
  not real citations in the original text.}

\begin{translationbib}
\item Donald E. Knuth. The \TeX book. Addison-Wesley, 1984. ISBN: 0-201-13448-9
\item Paul W. Abrahams, Karl Berry and Kathryn A. Hargreaves. \TeX\ for the
  Impatient. Addison-Wesley, 1990. ISBN: 0-201-51375-7
\item David Salomon. The advanced \TeX book.  New York : Springer, 1995. ISBN:0-387-94556-3
\end{translationbib}

\chapter{外文资料的调研阅读报告或书面翻译}

\title{英文资料的中文标题}

{\heiti 摘要:} 本章为外文资料翻译内容。如果有摘要可以直接写上来,这部分好像没有
明确的规定。

\section{单目标规划}
北冥有鱼,其名为鲲。鲲之大,不知其几千里也。化而为鸟,其名为鹏。鹏之背,不知其几
千里也。怒而飞,其翼若垂天之云。是鸟也,海运则将徙于南冥。南冥者,天池也。
\begin{equation}\tag*{(123)}
 p(y|\mathbf{x}) = \frac{p(\mathbf{x},y)}{p(\mathbf{x})}=
\frac{p(\mathbf{x}|y)p(y)}{p(\mathbf{x})}
\end{equation}

吾生也有涯,而知也无涯。以有涯随无涯,殆已!已而为知者,殆而已矣!为善无近名,为
恶无近刑,缘督以为经,可以保身,可以全生,可以养亲,可以尽年。

\subsection{线性规划}
庖丁为文惠君解牛,手之所触,肩之所倚,足之所履,膝之所倚,砉然响然,奏刀騞然,莫
不中音,合于桑林之舞,乃中经首之会。
\begin{table}[ht]
\centering
  \centering
  \caption*{表~1\hskip1em 这是手动编号但不出现在索引中的一个表格例子}
  \label{tab:badtabular3}
  \begin{tabular}[c]{|m{1.5cm}|c|c|c|c|c|c|}\hline
    \multicolumn{2}{|c|}{Network Topology} & \# of nodes &
    \multicolumn{3}{c|}{\# of clients} & Server \\\hline
    GT-ITM & Waxman Transit-Stub & 600 &
    \multirow{2}{2em}{2\%}&
    \multirow{2}{2em}{10\%}&
    \multirow{2}{2em}{50\%}&
    \multirow{2}{1.2in}{Max. Connectivity}\\\cline{1-3}
    \multicolumn{2}{|c|}{Inet-2.1} & 6000 & & & &\\\hline
    & \multicolumn{2}{c|}{ABCDEF} &\multicolumn{4}{c|}{} \\\hline
\end{tabular}
\end{table}

文惠君曰:“嘻,善哉!技盖至此乎?”庖丁释刀对曰:“臣之所好者道也,进乎技矣。始臣之
解牛之时,所见无非全牛者;三年之后,未尝见全牛也;方今之时,臣以神遇而不以目视,
官知止而神欲行。依乎天理,批大郤,导大窾,因其固然。技经肯綮之未尝,而况大坬乎!
良庖岁更刀,割也;族庖月更刀,折也;今臣之刀十九年矣,所解数千牛矣,而刀刃若新发
于硎。彼节者有间而刀刃者无厚,以无厚入有间,恢恢乎其于游刃必有余地矣。是以十九年
而刀刃若新发于硎。虽然,每至于族,吾见其难为,怵然为戒,视为止,行为迟,动刀甚微,
謋然已解,如土委地。提刀而立,为之而四顾,为之踌躇满志,善刀而藏之。”

文惠君曰:“善哉!吾闻庖丁之言,得养生焉。”


\subsection{非线性规划}
孔子与柳下季为友,柳下季之弟名曰盗跖。盗跖从卒九千人,横行天下,侵暴诸侯。穴室枢
户,驱人牛马,取人妇女。贪得忘亲,不顾父母兄弟,不祭先祖。所过之邑,大国守城,小
国入保,万民苦之。孔子谓柳下季曰:“夫为人父者,必能诏其子;为人兄者,必能教其弟。
若父不能诏其子,兄不能教其弟,则无贵父子兄弟之亲矣。今先生,世之才士也,弟为盗
跖,为天下害,而弗能教也,丘窃为先生羞之。丘请为先生往说之。”

柳下季曰:“先生言为人父者必能诏其子,为人兄者必能教其弟,若子不听父之诏,弟不受
兄之教,虽今先生之辩,将奈之何哉?且跖之为人也,心如涌泉,意如飘风,强足以距敌,
辩足以饰非。顺其心则喜,逆其心则怒,易辱人以言。先生必无往。”

孔子不听,颜回为驭,子贡为右,往见盗跖。

\subsection{整数规划}
盗跖乃方休卒徒大山之阳,脍人肝而餔之。孔子下车而前,见谒者曰:“鲁人孔丘,闻将军
高义,敬再拜谒者。”谒者入通。盗跖闻之大怒,目如明星,发上指冠,曰:“此夫鲁国之
巧伪人孔丘非邪?为我告之:尔作言造语,妄称文、武,冠枝木之冠,带死牛之胁,多辞缪
说,不耕而食,不织而衣,摇唇鼓舌,擅生是非,以迷天下之主,使天下学士不反其本,妄
作孝弟,而侥幸于封侯富贵者也。子之罪大极重,疾走归!不然,我将以子肝益昼餔之膳。”


\chapter{其它附录}
前面两个附录主要是给本科生做例子。其它附录的内容可以放到这里,当然如果你愿意,可
以把这部分也放到独立的文件中,然后将其到主文件中。
%本科生翻译论文
% % \end{appendix}

%%%%%%%%%%%%%%%%%%%%%%%%%%%%%%%%%%%
% 硕士研究生
%%%%%%%%%%%%%%%%%%%%%%%%%%%%%%%%%%%
% TBD

%%%%%%%%%%%%%%%%%%%%%%%%%%%%%%%%%%%
% 博士研究生
%%%%%%%%%%%%%%%%%%%%%%%%%%%%%%%%%%%
% TBD

%%%%%%%%%%%%%%%%%%%%%%%%%%%%%%%%%%%%%%%%%%%%%%%%%%%%%%%%%%%%
% \frontmatter
% % !Mode:: "TeX:UTF-8"

\hitsetup{
  %******************************
  % 注意:
  %   1. 配置里面不要出现空行
  %   2. 不需要的配置信息可以删除
  %******************************
  ctitleone={局部多孔质气体静压},%本科生封面使用
  ctitletwo={轴承关键技术的研究},%本科生封面使用
  ctitlecover={局部多孔质气体静压轴承关键技术的研究},%放在封面中使用,自由断行
  ctitle={局部多孔质气体静压轴承关键技术的研究},%放在原创性声明中使用
  csubtitle={一条副标题}, %一般情况没有,可以注释掉
  cxueke={工学},
  csubject={机械制造及其自动化},
  caffil={机电工程学院},
  cauthor={于冬梅},
  csupervisor={某某某教授},
  % 日期自动使用当前时间,若需指定按如下方式修改:
  cdate={2024年04月01日},
  cstudentid={1200300101},
}
 % 封面
% \makecover
% \input{front/denotation}%物理量名称表,符合规范为主,有要求添加
% %\cleardoublepage  自定义在什么位置进行右翻页
% \tableofcontents    % 中文目录
% %\cleardoublepage  自定义在什么位置进行右翻页
% \tableofengcontents % 英文目录,硕本不要求

% \mainmatter
% %\linenumbers %debug 选项
% %\layout %debug 选项
% %\floatdiagram %debug 选项
% %\begin{figure} %debug 选项
% %\currentfloat %debug 选项
% %\tryintextsep{\intextsep} %debug 选项
% %\trytopfigrule{0.5pt} %debug 选项
% %\trybotfigrule{1pt} %debug 选项
% %\setlayoutscale{0.9} %debug 选项
% %\floatdesign %debug 选项
% %\caption{Float layout with rules}\label{fig:fludf} %debug 选项
% %\end{figure} %debug 选项
% \include{body/introduction}

% \backmatter
% %硕博书序
% % % !Mode:: "TeX:UTF-8" 
\begin{conclusions}

学位论文的结论作为论文正文的最后一章单独排写,但不加章标题序号。

结论应是作者在学位论文研究过程中所取得的创新性成果的概要总结,不能与摘要混为一谈。博士学位论文结论应包括论文的主要结果、创新点、展望三部分,在结论中应概括论文的核心观点,明确、客观地指出本研究内容的创新性成果(含新见解、新观点、方法创新、技术创新、理论创新),并指出今后进一步在本研究方向进行研究工作的展望与设想。对所取得的创新性成果应注意从定性和定量两方面给出科学、准确的评价,分(1)、(2)、(3)…条列出,宜用“提出了”、“建立了”等词叙述。

\end{conclusions}
   % 结论
% % \bibliographystyle{hithesis} %如果没有参考文献时候
% % \bibliography{reference}
% % \begin{appendix}%附录
% % \input{back/appA.tex}
% % \end{appendix}
% % \include{back/publications}    % 所发文章
% % \include{back/ceindex}    % 索引, 根据自己的情况添加或者不添加,选择自动添加或者手工添加。
% % \authorization %授权
% % %\authorization[saomiao.pdf] %添加扫描页的命令,与上互斥
% % % !Mode:: "TeX:UTF-8"
\begin{acknowledgements}

衷心感谢导师~XXX~教授对本人的精心指导。他的言传身教将使我终生受益。

\end{acknowledgements}
 %致谢
% % \include{back/resume}          % 博士学位论文有个人简介

% %本科书序为:
% % !Mode:: "TeX:UTF-8" 
\begin{conclusions}

学位论文的结论作为论文正文的最后一章单独排写,但不加章标题序号。

结论应是作者在学位论文研究过程中所取得的创新性成果的概要总结,不能与摘要混为一谈。博士学位论文结论应包括论文的主要结果、创新点、展望三部分,在结论中应概括论文的核心观点,明确、客观地指出本研究内容的创新性成果(含新见解、新观点、方法创新、技术创新、理论创新),并指出今后进一步在本研究方向进行研究工作的展望与设想。对所取得的创新性成果应注意从定性和定量两方面给出科学、准确的评价,分(1)、(2)、(3)…条列出,宜用“提出了”、“建立了”等词叙述。

\end{conclusions}
   % 结论
% \bibliographystyle{hithesis}
% \bibliography{reference}
% \authorization %授权
% % \authorization[saomiao.pdf] %添加扫描页的命令,与上互斥
% % !Mode:: "TeX:UTF-8"
\begin{acknowledgements}

衷心感谢导师~XXX~教授对本人的精心指导。他的言传身教将使我终生受益。

\end{acknowledgements}
 %致谢
% \begin{appendix}%附录
%     \input{back/appendix01}%本科生翻译论文
% \end{appendix}

\end{document}

% Local Variables:
% TeX-engine: xetex
% End:
