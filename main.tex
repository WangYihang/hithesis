% !Mode:: "TeX:UTF-8"
% !Mode:: "TeX:UTF-8"
\documentclass[
    newtxmath=true,
    newgeometry=two,
    capcenterlast=true,
    subcapcenterlast=true,
    openright=false,
    library=false,
    absupper=true,
    fontset=windowsnew,
    type=bachelor,
    campus=harbin,
    progress=interim
]{hithesis}
% 此处选项中不要有空格
%%%%%%%%%%%%%%%%%%%%%%%%%%%%%%%%%%%%%%%%%%%%%%%%%%%%%%%%%%%%%%%%%%%%%%%%%%%%%%%%
% 必填选项
% type=doctor|master|bachelor
%%%%%%%%%%%%%%%%%%%%%%%%%%%%%%%%%%%%%%%%%%%%%%%%%%%%%%%%%%%%%%%%%%%%%%%%%%%%%%%%
% 选填选项(选填选项的缺省值已经尽可能满足了大多数需求,除非明确知道自己有什么
% 需求)
% campus=shenzhen|weihai|harbin
%   含义:校区选项,默认harbin
% progress=opening|interim|
%   含义:毕业设计的阶段
%       opening 开题
%       interim 中期
%       final 结题
% glue=true|false
%   含义:由于我工规范中要求字体行距在一个闭区间内,这个选项为true表示tex自
%   动选择,为false表示区间内一个最接近版心要求行数的要求的默认值,缺省值为
%   false。
% tocfour=true|false
%   含义:是否添加第四级目录,只对本科文科个别要求四级目录有效,缺省值为
%   false
% fontset=siyuan|windowsnew|windowsold
%   含义:注意这个选项视为了解决特殊问题而设置,比如用有些发行版本的linux排
%   版时可能(大多数发行版不会)会遇到的字体无法载入的问题,或者字体载入之
%   后出现无法复制的问题以及想要解决排版如 biang biang 面的 biang 这类中易
%   宋体无法识别的汉字的问题。没有特殊的需要不推荐使用这个选项。
%   
%   如果是安装了 windows 字体的 linux 系统,可以填写windowsnew(win vista
%   以后 的字体)或 windowsold(vista 以前)或者想用思源宋体并且是已经安装
%   了思源宋体的任何系统,填写siyuan选项。缺省值为空,自动识别系统并匹配字体
%   。模板版中给出的思源字体定义文件定义的思源字体的版本是Adobe版,其他字体
%   是windowsnew字体。
% tocblank=true|false
%   含义:目录中第一章之前,是否加一行空白。缺省值为true。
% chapterhang=true|false
%   含义:目录的章标题是否悬挂居中,规范中要求章标题少于15字,所以这个选项
%   有无没什么用,除了特殊需求。缺省值为true。
% fulltime=true|false
%   含义:是否全日制,缺省值为true。非全日制如同等学力等,要在cover中设置类
%   型,封面中不同格式
% subtitle=true|false
%   含义:论文题目是否含有副标题,缺省值为false,如果有要在cover中设置副标
%   题内容,封面中显示。
% newgeometry=one|two
%   含义:规范中的自相矛盾之处,版芯是否包含页眉页脚,旧方法是按照包含页眉
%   页脚来设置。该选项是多选选项,如果没有这个选项,缺省值是旧模板的版芯设
%   置方法,如果设置该选项one或two,分别对应两种页眉页码对应版芯线的相对位
%   置。第一种是严格按照规范要求,难看。第二种微调了页眉页码位置,好一点。
% debug=true|false
%   含义:是否显示版芯框和行号,用来调试。默认否。
% openright=true|false
%   含义:博士论文是否要求章节首页必须在奇数页,此选项不在规范要求中,按个
%   人喜好自行决定。 默认否。注意,窝工的默认情况是打印版博士论文要求右翻页
%   ,电子版要求非右翻页且无空白页。如果想DIY(或身不由己DIY)在什么地方右
%   翻页,将这个选项设置为false,然后在目标位置添加`\cleardoublepage`命令即
%   可。
% library=true|false
%   含义:是否为提交到图书馆的电子版。默认否。注意:如果设置成true,那么
%   openright选项将被强制转换为false。
% capcenterlast=true|false
%   含义:图题、表题最后一行是否居中对齐(我工规范要求居中,但不要求居中对
%   齐),此选项不在规范要求中,按个人喜好自行决定。默认否。
% subcapcenterlast=true|false
%   含义:子图图题最后一行是否居中对齐(我工规范要求居中,但不要求居中对齐
%   ),此选项不在规范要求中,按个人喜好自行决定。默认否。
% absupper=true|false
%   含义:中文目录中的英文摘要在中文目录中的大小写样式歧义,在规范中要求首
%   字母大写,在work样例中是全大写。该选项控制是否全大写。默认否。
% bsmainpagenumberline=true|false
%   含义:由于本科生论文官方模板的页码和页眉格式混乱,提供这个选项自定义设
%   置是否在正文中显示页码横线,默认否。
% bsfrontpagenumberline=true|false
%   含义:由于本科生论文官方模板的页码和页眉格式混乱,提供这个选项自定义设
%   置是否在前文中显示页码横线,默认否。
% bsheadrule=true|false
%   含义:由于本科生论文官方模板的页码和页眉格式混乱,提供这个选项自定义设
%   置是否显示页眉横线,默认显示。
% splitbibitem=true|false
%   含义:参考文献每一个条目内能不能断页,应广大刀客要求添加。默认否。
% newtxmath=true|false
%   含义:数学字体是否使用新罗马。默认是。
%%%%%%%%%%%%%%%%%%%%%%%%%%%%%%%%%%%%%%%%%%%%%%%%%%%%%%%%%%%%%%%%%%%%%%%%%%%%%%%%

\usepackage{hithesis}

% 定义封面信息
% !Mode:: "TeX:UTF-8"

\hitsetup{
  %******************************
  % 注意:
  %   1. 配置里面不要出现空行
  %   2. 不需要的配置信息可以删除
  %******************************
  ctitleone={局部多孔质气体静压},%本科生封面使用
  ctitletwo={轴承关键技术的研究},%本科生封面使用
  ctitlecover={局部多孔质气体静压轴承关键技术的研究},%放在封面中使用,自由断行
  ctitle={局部多孔质气体静压轴承关键技术的研究},%放在原创性声明中使用
  csubtitle={一条副标题}, %一般情况没有,可以注释掉
  cxueke={工学},
  csubject={机械制造及其自动化},
  caffil={机电工程学院},
  cauthor={于冬梅},
  csupervisor={某某某教授},
  % 日期自动使用当前时间,若需指定按如下方式修改:
  cdate={2024年04月01日},
  cstudentid={1200300101},
}

% 定义所包含图片根目录
\graphicspath{{figures/}}
% 物理量名称表,学院有要求添加
% \input{front/denotation}


\begin{document}
%%%%%%%%%%%%%%%%%%%%%%%%%%%%%%%%%%%
% 本科生开题答辩
%%%%%%%%%%%%%%%%%%%%%%%%%%%%%%%%%%%
% TBD

%%%%%%%%%%%%%%%%%%%%%%%%%%%%%%%%%%%
% 本科生中期答辩
%%%%%%%%%%%%%%%%%%%%%%%%%%%%%%%%%%%
\frontmatter
% 生成封面
\makecover
% 生成目录
\tableofcontents

\mainmatter
% 根据学校教务处中期报告模版将内容分为如下几个章节
% !Mode:: "TeX:UTF-8"

\chapter{课题研究内容}

本课题来源于...项目。
感谢前辈们 \cite{Chen1992} 的贡献。
\chapter{研究方案}

本课题采用...的研究方案。

\chapter{目前已完成的研究工作及成果}

% \section{开题时工作安排时间表}
% 开题时,对毕业设计的进度与时间节点安排如表 \ref{tab:schedule} 所示,\emph{目前正在按照计划进行}。
% \begin{table}[htbp]
%     \centering
%     \bicaption{}{开题时工作安排时间表}{Table$\!$}{Schedule in opening report}\vspace{0.5em}\wuhao
%     \begin{tabularx}{\textwidth}{lX}
%     \toprule[1.5pt]
%     预计时间 & 计划完成的工作 \\
%     \midrule[1pt]
%     2024.01 & 工作 1 \\
%     2024.02 & 工作 2 \\
%     2024.03 & 工作 3 \\
%     2024.03 & 工作 4 \\
%     2024.04 & 工作 5 \\
%     2024.05 & 工作 6 \\
%     2024.06 & 工作 7 \\
%     \bottomrule[1.5pt]
%     \label{tab:schedule}
%     \end{tabularx}
% \end{table}

下面对已完成的成果进行介绍。
\chapter{后期拟完成的研究工作与进度安排}\label{ch:future_work}

\section{后期拟完成的研究工作}
    \begin{enumerate}
        \item 工作 1
        \item 工作 2
        \item 工作 3
        \item 工作 4
        \item 工作 5
        \item 工作 6
    \end{enumerate}

\section{后期工作时间安排表}
    \begin{table}[htbp]
        \centering
        \bicaption{}{后期工作安排时间表}{Table$\!$}{Schedule in future}\vspace{0.5em}\wuhao
        \begin{tabularx}{\textwidth}{llX}
        \toprule[1.5pt]
        预计时间 & 周数 & 计划完成的工作 \\
        \midrule[1pt]
        2024.03 & 2 & 工作 1 \\
        2024.03 & 2 & 工作 2 \\
        2024.04 & 2 & 工作 3 \\
        2024.04 & 2 & 工作 4 \\
        2024.05 & 2 & 工作 5 \\
        2024.06 & 2 & 工作 6 \\
        \bottomrule[1.5pt]
        \label{tab:futureschedule}
        \end{tabularx}
    \end{table}




\chapter{存在的困难与问题}

\begin{enumerate}
    \item 困难与问题 1
    \item 困难与问题 2
    \item 困难与问题 3
\end{enumerate}
\chapter{如期完成全部论文工作的可能性}

可以如期完成

\backmatter
\bibliographystyle{hithesis}
\bibliography{reference}

%%%%%%%%%%%%%%%%%%%%%%%%%%%%%%%%%%%
% 本科生毕业设计(论文)
%%%%%%%%%%%%%%%%%%%%%%%%%%%%%%%%%%%
% TBD


%%%%%%%%%%%%%%%%%%%%%%%%%%%%%%%%%%%
% 硕士研究生
%%%%%%%%%%%%%%%%%%%%%%%%%%%%%%%%%%%
% TBD

%%%%%%%%%%%%%%%%%%%%%%%%%%%%%%%%%%%
% 博士研究生
%%%%%%%%%%%%%%%%%%%%%%%%%%%%%%%%%%%
% TBD

%%%%%%%%%%%%%%%%%%%%%%%%%%%%%%%%%%%%%%%%%%%%%%%%%%%%%%%%%%%%
% \frontmatter
% % !Mode:: "TeX:UTF-8"

\hitsetup{
  %******************************
  % 注意:
  %   1. 配置里面不要出现空行
  %   2. 不需要的配置信息可以删除
  %******************************
  ctitleone={局部多孔质气体静压},%本科生封面使用
  ctitletwo={轴承关键技术的研究},%本科生封面使用
  ctitlecover={局部多孔质气体静压轴承关键技术的研究},%放在封面中使用,自由断行
  ctitle={局部多孔质气体静压轴承关键技术的研究},%放在原创性声明中使用
  csubtitle={一条副标题}, %一般情况没有,可以注释掉
  cxueke={工学},
  csubject={机械制造及其自动化},
  caffil={机电工程学院},
  cauthor={于冬梅},
  csupervisor={某某某教授},
  % 日期自动使用当前时间,若需指定按如下方式修改:
  cdate={2024年04月01日},
  cstudentid={1200300101},
}
 % 封面
% \makecover
% \input{front/denotation}%物理量名称表,符合规范为主,有要求添加
% %\cleardoublepage  自定义在什么位置进行右翻页
% \tableofcontents    % 中文目录
% %\cleardoublepage  自定义在什么位置进行右翻页
% \tableofengcontents % 英文目录,硕本不要求

% \mainmatter
% %\linenumbers %debug 选项
% %\layout %debug 选项
% %\floatdiagram %debug 选项
% %\begin{figure} %debug 选项
% %\currentfloat %debug 选项
% %\tryintextsep{\intextsep} %debug 选项
% %\trytopfigrule{0.5pt} %debug 选项
% %\trybotfigrule{1pt} %debug 选项
% %\setlayoutscale{0.9} %debug 选项
% %\floatdesign %debug 选项
% %\caption{Float layout with rules}\label{fig:fludf} %debug 选项
% %\end{figure} %debug 选项
% \include{body/introduction}

% \backmatter
% %硕博书序
% % % !Mode:: "TeX:UTF-8" 
\begin{conclusions}

学位论文的结论作为论文正文的最后一章单独排写,但不加章标题序号。

结论应是作者在学位论文研究过程中所取得的创新性成果的概要总结,不能与摘要混为一谈。博士学位论文结论应包括论文的主要结果、创新点、展望三部分,在结论中应概括论文的核心观点,明确、客观地指出本研究内容的创新性成果(含新见解、新观点、方法创新、技术创新、理论创新),并指出今后进一步在本研究方向进行研究工作的展望与设想。对所取得的创新性成果应注意从定性和定量两方面给出科学、准确的评价,分(1)、(2)、(3)…条列出,宜用“提出了”、“建立了”等词叙述。

\end{conclusions}
   % 结论
% % \bibliographystyle{hithesis} %如果没有参考文献时候
% % \bibliography{reference}
% % \begin{appendix}%附录
% % \input{back/appA.tex}
% % \end{appendix}
% % \include{back/publications}    % 所发文章
% % \include{back/ceindex}    % 索引, 根据自己的情况添加或者不添加,选择自动添加或者手工添加。
% % \authorization %授权
% % %\authorization[saomiao.pdf] %添加扫描页的命令,与上互斥
% % % !Mode:: "TeX:UTF-8"
\begin{acknowledgements}

衷心感谢导师~XXX~教授对本人的精心指导。他的言传身教将使我终生受益。

\end{acknowledgements}
 %致谢
% % \include{back/resume}          % 博士学位论文有个人简介

% %本科书序为:
% % !Mode:: "TeX:UTF-8" 
\begin{conclusions}

学位论文的结论作为论文正文的最后一章单独排写,但不加章标题序号。

结论应是作者在学位论文研究过程中所取得的创新性成果的概要总结,不能与摘要混为一谈。博士学位论文结论应包括论文的主要结果、创新点、展望三部分,在结论中应概括论文的核心观点,明确、客观地指出本研究内容的创新性成果(含新见解、新观点、方法创新、技术创新、理论创新),并指出今后进一步在本研究方向进行研究工作的展望与设想。对所取得的创新性成果应注意从定性和定量两方面给出科学、准确的评价,分(1)、(2)、(3)…条列出,宜用“提出了”、“建立了”等词叙述。

\end{conclusions}
   % 结论
% \bibliographystyle{hithesis}
% \bibliography{reference}
% \authorization %授权
% % \authorization[saomiao.pdf] %添加扫描页的命令,与上互斥
% % !Mode:: "TeX:UTF-8"
\begin{acknowledgements}

衷心感谢导师~XXX~教授对本人的精心指导。他的言传身教将使我终生受益。

\end{acknowledgements}
 %致谢
% \begin{appendix}%附录
%     \input{back/appendix01}%本科生翻译论文
% \end{appendix}

\end{document}

% Local Variables:
% TeX-engine: xetex
% End:
